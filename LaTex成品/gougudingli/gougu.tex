\documentclass[UTF8]{ctexart}
\usepackage{titling}
\title{杂谈勾股定理}
\author{啊飘}
\date{\today}
\bibliographystyle{plain}
\begin{document}
\maketitle
\tableofcontents
\section{勾股定理在古代}
西方称勾股定理为毕达哥拉斯定理,将勾股定理的发现归功于公元前6世纪的毕达哥拉斯学派[1]。该学派得到了一个法则,可以求出可排成直角三角形三边的三元数组。毕达哥拉斯学派没有书面著作,该定理的严格表述和证明则见于欧几里德《几何原本》的命题47:“直角三角形斜边上的正方形等于两直角边上的两个正方形之和。”证明是用面积做的。

我国《周髀算经》载商高(约公元前12 世纪)答周公问:
\begin{quote}
	勾广三,股修四,径隅五。
\end{quote}
又载陈子(约公元前7-6 世纪) 答荣方问:
\begin{quote}
若求邪至日者,以日下为勾,日高为股,勾股各自乘,并而开方除之,得邪至日。
\end{quote}
都较古希腊更早。后者已经明确道出勾股定理的一般形式。图1是我国古代对勾股定理的一种证明。
\newpage
\section{勾股定理的近代形式}
勾股定理可以用现代语言表述如下:

定理1(勾股定理) 直角三角形斜边的平方等于两
腰的平方和.

可以用符号语言表述为: 设直角三角形ABC,其中$\angle$C=$90^\circ$,则有
\begin{equation}
BC^2+AC^2=AB^2
\end{equation}
满足式(1)的整数称为勾股教。第1节所说毕达哥拉斯学派得到的三元数组就县是数组就是勾股数。下表列出一些较小的勾股数:

\begin{tabular}{|rrr|}
\hline
直角边 $a$ & 直角边 $b$ & 斜边 $c$ \\
\hline
   3&   4&   5\\
   5&   12&   13\\
\hline
\end{tabular}%
\qquad
($a^2+b^2=C^2$)
\end{document}